\documentclass[xcolor=dvipsnames]{beamer}
%\documentclass[xcolor=svgnames]{beamer} 
\usepackage{epsfig}
\usepackage{graphicx}
\usepackage{epsfig}
\usepackage{subfigure}
\usepackage{hyperref}
\usepackage[spanish,activeacute]{babel}
\usepackage{fancybox}
\usepackage{helvet} %este guta
\usepackage{multirow}
\usepackage{rotating}
\usepackage{relsize}

%\hypersetup{
%    pdffitwindow=true,      % page fit to window when opened
%    pdftitle={My title},    % title
%    pdfnewwindow=true,      % links in new window
%    colorlinks=true,       % false: boxed links; true: colored links
%    linkcolor=purple,          % color of internal links
%    citecolor=green,        % color of links to bibliography
%    filecolor=magenta,      % color of file links
%    urlcolor=cyan           % color of external links
%}

%\usetheme{Copenhagen}  % name of theme substututed here
%\usetheme{Luebeck}
%\usetheme{Malmoe} % Madrid y Malmoe van bien con infolines
%\usetheme{default}
\usetheme{Madrid} % default minimalista
\usecolortheme{rose}
\useoutertheme{infolines} 
\setbeamertemplate{navigation symbols}{}

% changing colors
\setbeamercolor*{palette primary}{use=structure,fg=white,bg=NavyBlue}
\setbeamercolor*{palette secondary}{use=structure,fg=white,bg=CornflowerBlue}
\setbeamercolor*{palette tertiary}{use=structure,fg=white,bg=SpringGreen}

% changing rounded blocks,comment to rounded and shadowed blocks
\setbeamertemplate{blocks}[default]

% define colors
%\usecolortheme[RGB={205,173,0}]{structure} 
%\usecolortheme[RGB={154,144,144}]{structure} %gray
%\definecolor{mypink}{rgb}{0.7,0.2,0.3}
%\definecolor{mynavyblue}{rgb}{0.08,0.06,0.4}

%%%%%%%%%% pone el título de los frames al centro
%\setbeamercolor{frametitle}{fg=blue}
%\setbeamerfont{frametitle}{series=\bfseries}
%\setbeamertemplate{frametitle}
%{
%\begin{centering}
%\insertframetitle\par
%\end{centering}
%}
%%%%%%%%%%%%%%%%%%%%%%%%%%%%%%%%%%%%%%%%%%%%%%%%%%%%

\begin{document}
%\includeonlyframes{ggplot2}
\title{Beamer Sample}
\subtitle{Based on Beamer version 3.06}
\author{Blanca Vargas Govea}
\institute[CENIDET]{Departamento de Ciencias Computacionales\\CENIDET}
\date{\today}

\begin{frame}[plain] 
  \titlepage
\end{frame}
%-------------------------------------
\section*{Contenido}
\begin{frame}{Contenido}
  %\color{mynavyblue}\textbf{Texto}
  \tableofcontents%[pausesections]
\end{frame}
%-------------------------------------
\section{Secci'on 1}
\begin{frame}{Introduction}
Intro
\url{http://www.bookcrossing.com/}
\end{frame}

%-------------------------------------
\section{Secci'on 2}

\begin{frame}{Bloque}
\begin{block}{Introduction to {\LaTeX}}
"Beamer is a {\LaTeX}class for creating presentations
that are held using a projector..."
\end{block}

\begin{example}{Introduction to {\LaTeX}}
"Beamer is a {\LaTeX}class for creating presentations
that are held using a projector..."
\end{example}

\begin{alertblock}{Introduction to {\LaTeX}}
"Beamer is a {\LaTeX}class for creating presentations
that are held using a projector..."
\end{alertblock}

\end{frame}
%-------------------------------------
%\bibliographystyle{apalike2}
%\bibliography{evan}
\end{document}
%%%%%% ejemplos

%\begin{block}{Introduction to {\LaTeX}}
%"Beamer is a {\LaTeX}class for creating presentations
%that are held using a projector..."
%\end{block}

%\begin{example}{Introduction to {\LaTeX}}
%"Beamer is a {\LaTeX}class for creating presentations
%that are held using a projector..."
%\end{example}

%\begin{alertblock}{Introduction to {\LaTeX}}
%"Beamer is a {\LaTeX}class for creating presentations
%that are held using a projector..."
%\end{alertblock}

%\begin{columns}[t]
%\column{.5\textwidth}
%\begin{block}{Column 1 Header}
%Column 1 Body Text
%\end{block}
%\column{.5\textwidth}
%\begin{block}{Column 2 Header}
%Column 2 Body Text
%\end{block}
%\end{columns}

%\shadowbox{Sample Text}
%\fbox{Sample Text}
%\doublebox{Sample Text}
%\ovalbox{Sample Text}
%\Ovalbox{Sample Text}

%\section*{Proyecto postdoctoral} * para que no aparezca en el contenido
